\documentclass[10pt,a4paper,onecolumn]{article}
\usepackage{marginnote}
\usepackage{graphicx}
\usepackage{xcolor}
\usepackage{authblk,etoolbox}
\usepackage{titlesec}
\usepackage{calc}
\usepackage{tikz}
\usepackage{hyperref}
\hypersetup{colorlinks,breaklinks=true,
            urlcolor=[rgb]{0.0, 0.5, 1.0},
            linkcolor=[rgb]{0.0, 0.5, 1.0}}
\usepackage{caption}
\usepackage{tcolorbox}
\usepackage{amssymb,amsmath}
\usepackage{ifxetex,ifluatex}
\usepackage{seqsplit}
\usepackage{xstring}

\usepackage{float}
\let\origfigure\figure
\let\endorigfigure\endfigure
\renewenvironment{figure}[1][2] {
    \expandafter\origfigure\expandafter[H]
} {
    \endorigfigure
}


\usepackage{fixltx2e} % provides \textsubscript
\usepackage[
  backend=biber,
%  style=alphabetic,
%  citestyle=numeric
]{biblatex}
\bibliography{paper.bib}

% --- Splitting \texttt --------------------------------------------------

\let\textttOrig=\texttt
\def\texttt#1{\expandafter\textttOrig{\seqsplit{#1}}}
\renewcommand{\seqinsert}{\ifmmode
  \allowbreak
  \else\penalty6000\hspace{0pt plus 0.02em}\fi}


% --- Pandoc does not distinguish between links like [foo](bar) and
% --- [foo](foo) -- a simplistic Markdown model.  However, this is
% --- wrong:  in links like [foo](foo) the text is the url, and must
% --- be split correspondingly.
% --- Here we detect links \href{foo}{foo}, and also links starting
% --- with https://doi.org, and use path-like splitting (but not
% --- escaping!) with these links.
% --- Another vile thing pandoc does is the different escaping of
% --- foo and bar.  This may confound our detection.
% --- This problem we do not try to solve at present, with the exception
% --- of doi-like urls, which we detect correctly.


\makeatletter
\let\href@Orig=\href
\def\href@Urllike#1#2{\href@Orig{#1}{\begingroup
    \def\Url@String{#2}\Url@FormatString
    \endgroup}}
\def\href@Notdoi#1#2{\def\tempa{#1}\def\tempb{#2}%
  \ifx\tempa\tempb\relax\href@Urllike{#1}{#2}\else
  \href@Orig{#1}{#2}\fi}
\def\href#1#2{%
  \IfBeginWith{#1}{https://doi.org}%
  {\href@Urllike{#1}{#2}}{\href@Notdoi{#1}{#2}}}
\makeatother

\newlength{\cslhangindent}
\setlength{\cslhangindent}{1.5em}
\newlength{\csllabelwidth}
\setlength{\csllabelwidth}{3em}
\newenvironment{CSLReferences}[3] % #1 hanging-ident, #2 entry spacing
 {% don't indent paragraphs
  \setlength{\parindent}{0pt}
  % turn on hanging indent if param 1 is 1
  \ifodd #1 \everypar{\setlength{\hangindent}{\cslhangindent}}\ignorespaces\fi
  % set entry spacing
  \ifnum #2 > 0
  \setlength{\parskip}{#2\baselineskip}
  \fi
 }%
 {}
\usepackage{calc}
\newcommand{\CSLBlock}[1]{#1\hfill\break}
\newcommand{\CSLLeftMargin}[1]{\parbox[t]{\csllabelwidth}{#1}}
\newcommand{\CSLRightInline}[1]{\parbox[t]{\linewidth - \csllabelwidth}{#1}}
\newcommand{\CSLIndent}[1]{\hspace{\cslhangindent}#1}

% --- Page layout -------------------------------------------------------------
\usepackage[top=3.5cm, bottom=3cm, right=1.5cm, left=1.0cm,
            headheight=2.2cm, reversemp, includemp, marginparwidth=4.5cm]{geometry}

% --- Default font ------------------------------------------------------------
\renewcommand\familydefault{\sfdefault}

% --- Style -------------------------------------------------------------------
\renewcommand{\bibfont}{\small \sffamily}
\renewcommand{\captionfont}{\small\sffamily}
\renewcommand{\captionlabelfont}{\bfseries}

% --- Section/SubSection/SubSubSection ----------------------------------------
\titleformat{\section}
  {\normalfont\sffamily\Large\bfseries}
  {}{0pt}{}
\titleformat{\subsection}
  {\normalfont\sffamily\large\bfseries}
  {}{0pt}{}
\titleformat{\subsubsection}
  {\normalfont\sffamily\bfseries}
  {}{0pt}{}
\titleformat*{\paragraph}
  {\sffamily\normalsize}


% --- Header / Footer ---------------------------------------------------------
\usepackage{fancyhdr}
\pagestyle{fancy}
\fancyhf{}
%\renewcommand{\headrulewidth}{0.50pt}
\renewcommand{\headrulewidth}{0pt}
\fancyhead[L]{\hspace{-0.75cm}\includegraphics[width=5.5cm]{./img/logo.png}}
\fancyhead[C]{}
\fancyhead[R]{}
\renewcommand{\footrulewidth}{0.25pt}

\fancyfoot[L]{\parbox[t]{0.98\headwidth}{\footnotesize{\sffamily ¿citation\_author?, (2021). IC-Layout
Render: Image rendering tool for integrated circuit layout in
Python. \textit{Journal of Open Source Software}, ¿VOL?(¿ISSUE?), ¿PAGE?. \url{https://doi.org/DOI unavailable}}}}


\fancyfoot[R]{\sffamily \thepage}
\makeatletter
\let\ps@plain\ps@fancy
\fancyheadoffset[L]{4.5cm}
\fancyfootoffset[L]{4.5cm}

% --- Macros ---------

\definecolor{linky}{rgb}{0.0, 0.5, 1.0}

\newtcolorbox{repobox}
   {colback=red, colframe=red!75!black,
     boxrule=0.5pt, arc=2pt, left=6pt, right=6pt, top=3pt, bottom=3pt}

\newcommand{\ExternalLink}{%
   \tikz[x=1.2ex, y=1.2ex, baseline=-0.05ex]{%
       \begin{scope}[x=1ex, y=1ex]
           \clip (-0.1,-0.1)
               --++ (-0, 1.2)
               --++ (0.6, 0)
               --++ (0, -0.6)
               --++ (0.6, 0)
               --++ (0, -1);
           \path[draw,
               line width = 0.5,
               rounded corners=0.5]
               (0,0) rectangle (1,1);
       \end{scope}
       \path[draw, line width = 0.5] (0.5, 0.5)
           -- (1, 1);
       \path[draw, line width = 0.5] (0.6, 1)
           -- (1, 1) -- (1, 0.6);
       }
   }

% --- Title / Authors ---------------------------------------------------------
% patch \maketitle so that it doesn't center
\patchcmd{\@maketitle}{center}{flushleft}{}{}
\patchcmd{\@maketitle}{center}{flushleft}{}{}
% patch \maketitle so that the font size for the title is normal
\patchcmd{\@maketitle}{\LARGE}{\LARGE\sffamily}{}{}
% patch the patch by authblk so that the author block is flush left
\def\maketitle{{%
  \renewenvironment{tabular}[2][]
    {\begin{flushleft}}
    {\end{flushleft}}
  \AB@maketitle}}
\makeatletter
\renewcommand\AB@affilsepx{ \protect\Affilfont}
%\renewcommand\AB@affilnote[1]{{\bfseries #1}\hspace{2pt}}
\renewcommand\AB@affilnote[1]{{\bfseries #1}\hspace{3pt}}
\renewcommand{\affil}[2][]%
   {\newaffiltrue\let\AB@blk@and\AB@pand
      \if\relax#1\relax\def\AB@note{\AB@thenote}\else\def\AB@note{#1}%
        \setcounter{Maxaffil}{0}\fi
        \begingroup
        \let\href=\href@Orig
        \let\texttt=\textttOrig
        \let\protect\@unexpandable@protect
        \def\thanks{\protect\thanks}\def\footnote{\protect\footnote}%
        \@temptokena=\expandafter{\AB@authors}%
        {\def\\{\protect\\\protect\Affilfont}\xdef\AB@temp{#2}}%
         \xdef\AB@authors{\the\@temptokena\AB@las\AB@au@str
         \protect\\[\affilsep]\protect\Affilfont\AB@temp}%
         \gdef\AB@las{}\gdef\AB@au@str{}%
        {\def\\{, \ignorespaces}\xdef\AB@temp{#2}}%
        \@temptokena=\expandafter{\AB@affillist}%
        \xdef\AB@affillist{\the\@temptokena \AB@affilsep
          \AB@affilnote{\AB@note}\protect\Affilfont\AB@temp}%
      \endgroup
       \let\AB@affilsep\AB@affilsepx
}
\makeatother
\renewcommand\Authfont{\sffamily\bfseries}
\renewcommand\Affilfont{\sffamily\small\mdseries}
\setlength{\affilsep}{1em}


\ifnum 0\ifxetex 1\fi\ifluatex 1\fi=0 % if pdftex
  \usepackage[T1]{fontenc}
  \usepackage[utf8]{inputenc}

\else % if luatex or xelatex
  \ifxetex
    \usepackage{mathspec}
    \usepackage{fontspec}

  \else
    \usepackage{fontspec}
  \fi
  \defaultfontfeatures{Ligatures=TeX,Scale=MatchLowercase}

\fi
% use upquote if available, for straight quotes in verbatim environments
\IfFileExists{upquote.sty}{\usepackage{upquote}}{}
% use microtype if available
\IfFileExists{microtype.sty}{%
\usepackage{microtype}
\UseMicrotypeSet[protrusion]{basicmath} % disable protrusion for tt fonts
}{}

\usepackage{hyperref}
\hypersetup{unicode=true,
            pdftitle={IC-Layout Render: Image rendering tool for integrated circuit layout in Python},
            pdfborder={0 0 0},
            breaklinks=true}
\urlstyle{same}  % don't use monospace font for urls

% --- We redefined \texttt, but in sections and captions we want the
% --- old definition
\let\addcontentslineOrig=\addcontentsline
\def\addcontentsline#1#2#3{\bgroup
  \let\texttt=\textttOrig\addcontentslineOrig{#1}{#2}{#3}\egroup}
\let\markbothOrig\markboth
\def\markboth#1#2{\bgroup
  \let\texttt=\textttOrig\markbothOrig{#1}{#2}\egroup}
\let\markrightOrig\markright
\def\markright#1{\bgroup
  \let\texttt=\textttOrig\markrightOrig{#1}\egroup}


\usepackage{graphicx,grffile}
\makeatletter
\def\maxwidth{\ifdim\Gin@nat@width>\linewidth\linewidth\else\Gin@nat@width\fi}
\def\maxheight{\ifdim\Gin@nat@height>\textheight\textheight\else\Gin@nat@height\fi}
\makeatother
% Scale images if necessary, so that they will not overflow the page
% margins by default, and it is still possible to overwrite the defaults
% using explicit options in \includegraphics[width, height, ...]{}
\setkeys{Gin}{width=\maxwidth,height=\maxheight,keepaspectratio}
\IfFileExists{parskip.sty}{%
\usepackage{parskip}
}{% else
\setlength{\parindent}{0pt}
\setlength{\parskip}{6pt plus 2pt minus 1pt}
}
\setlength{\emergencystretch}{3em}  % prevent overfull lines
\providecommand{\tightlist}{%
  \setlength{\itemsep}{0pt}\setlength{\parskip}{0pt}}
\setcounter{secnumdepth}{0}
% Redefines (sub)paragraphs to behave more like sections
\ifx\paragraph\undefined\else
\let\oldparagraph\paragraph
\renewcommand{\paragraph}[1]{\oldparagraph{#1}\mbox{}}
\fi
\ifx\subparagraph\undefined\else
\let\oldsubparagraph\subparagraph
\renewcommand{\subparagraph}[1]{\oldsubparagraph{#1}\mbox{}}
\fi

\title{IC-Layout Render: Image rendering tool for integrated circuit
layout in Python}

        \author[1, 2]{João R. R. O. Martins}
         \author[1, 2]{Pietro M. Ferreira}
    
      \affil[1]{Université Paris-Saclay, CentraleSupélec, CNRS, Lab. de
Génie Électrique et Électronique de Paris,91192,Gif-sur-Yvette,France}

      \affil[2]{Sorbonne Université, CNRS, Lab. de Génie Électrique et
Électronique de Paris, 75252, Paris, France}
  \date{\vspace{-7ex}}

\begin{document}
\maketitle

\marginpar{

  \begin{flushleft}
  %\hrule
  \sffamily\small

  {\bfseries DOI:} \href{https://doi.org/10.5281/zenodo.5618268}{\color{linky}{10.5281/zenodo.5618268}}

  \vspace{2mm}

  {\bfseries Software}
  \begin{itemize}
    \setlength\itemsep{0em}
    \item \href{https://github.com/DrPiBlacksmith/icLayoutRender/releases/tag/Beta}{\color{linky}{Release}} \ExternalLink
    \item \href{https://github.com/DrPiBlacksmith/icLayoutRender}{\color{linky}{Repository}} \ExternalLink
    %\item \href{DOI unavailable}{\color{linky}{Archive}} \ExternalLink
  \end{itemize}

  \vspace{2mm}

  \par\noindent\hrulefill\par

  \vspace{2mm}

%  {\bfseries Editor:} \href{https://example.com}{Pending Editor} \ExternalLink \\
%  \vspace{1mm}
%    {\bfseries Reviewers:}
%  \begin{itemize}
%  \setlength\itemsep{0em}
%    \item \href{https://github.com/Pending Reviewers}{@Pending Reviewers}
%    \end{itemize}
%    \vspace{2mm}

  {\bfseries Submitted:} N/A\\
  {\bfseries Published:} N/A

  \vspace{2mm}
  {\bfseries License}\\
  Authors of papers retain copyright and release the work under a Creative Commons Attribution 4.0 International License (\href{http://creativecommons.org/licenses/by/4.0/}{\color{linky}{CC BY 4.0}}).

    \vspace{4mm}
  \end{flushleft}
}

\hypertarget{summary}{%
\section{Summary}\label{summary}}

Graphic Design System (GDSII) from (Calma Company, 1987) is a common
database file format to stream out integrated circuit (IC) masks, also
called layout, before fabrication. Being an essential part of circuit
development, designers are often familiarized in how to generate this
binary file on computer-aided design (CAD) tools. This output file
contains planar geometric shapes which represent physical 3D layers from
a specific process design kit (PDK). Commercial PDK usually presents
more than 50 layers, which are 2D visualized in CAD tools.

Reading a layout in those CAD tools is not a simple task, however
high-quality graphical image is available in such tool. Thus, an IC
designer is capable to understand the data representation and point out
improvements for the IC. Nevertheless, IC documentation is often
delivered in a lower quality 2D image depicted in a PDF file. Once the
PDF is generated; the circuit layout becomes hardly readable even for
experienced IC designers.

Scientific communications require accurate and repeatable results to be
considered prior publication. In microelectronics research field, a
layout picture is mandatory. While illustrations should be a vectorial
graph like, IC layout is often depicted from a low-quality bitmap
obtained from a screenshot or an image saving through the CAD tool. In
this scenario, having an accurate, readable, and reproducible layout
result is a challenge. Most publications illustrate IC layouts in lower
standards than the other illustration results, which hinds the required
physical solution to address state-of-the-art IC performance.

The tool icLayoutRender aims to a user-friendly transcription of a GDSII
file in a pdf file. Developed in Python 3, the image rendering requires
an input GDSII file \textless cellName.gds\textgreater{} and the PDK
layer color file \textless LayerColor\_PDK.map\textgreater{} to produce
an output PDF file \textless cellName.pdf\textgreater. The GDSII layers
will be rendered using the available PDK layer colors. Missing layers
will be neglected.

The tool icLayoutRender does: 1. confirm the entered values; 2. item
convert the \emph{.gds to a }.tex file; 3. call LuaTeX or pdflatex to
generate a \emph{.aux, and }.pdf.

The tool icLayoutRender requires: 1. Python3 installed with devel
options; 2. gdspy, pandas, math, and GDSLatexConverter (Vollmer, 2020)
Python libraries installed; 3. TexLive for a Linux installation, or
MikTEX for a Windows installation; 4. tikz LaTEX package to compile the
output *.pdf file.

Several examples using icLayoutRender are freely available on the
official website. Layer color file generation is also explained in the
user manual, as installation procedure and the tool operation in Linux
and Windows. The icLayoutRender image rendering tool has been used in
recent scientific communication (Martins et al., 2021) and the
illustration readability is remarkable.

\hypertarget{overview-of-iclayoutrender-tool}{%
\section{Overview of icLayoutRenderOTA_Miller.pdf

tool}\label{overview-of-iclayoutrender-tool}}

In commercial CAD tools, the GDSII file is obtained following a common
procedure File-\textgreater Export-\textgreater Stream. The output
stream file should be named as \textless cellName.gds\textgreater{} and
selected as the top cell to be exported from an available layout. The
GDSII file should be saved in the same folder as the proposed tool for a
user-friendly operation.

The LayerColors.map file is created for a specific PDK. Commercially
available PDK, as (\textbf{XFAB:019?}), has the layers: diffusion in
lime, poly-Si in red, n-type implantation in gold, p-type implantation
in pink, and others. The layer color file format is depicted in the
\autoref{ver:LayerMAP} One may notice that lime, red, gold, and pink
colors are represented in RGB color code. GDS layer number and name are
available in the PDK layer map file (see @ref\{fig:LayerNumbers\}),
while the color and its code are obtained in the technology file. A
user-friendly layer window (LSW) is often available aid both files
translation in the requested LayerColors.map. One may implement an
automation tool for such translation. However, this procedure is only
run once per PDK. GDS number, layer name, and color do not change
between different PDK versions. Moreover, CAD tools usually uses the
color code proposed in the example \autoref{ver:LayerMAP}. Thus, this
procedure is only required in the installation of a new PDK and the
layer colors are usually the same. The GDS number is the data that
mostly change between different PDK files.
\includegraphics{img/LayerNumbers.png}%[Checking layer numbers in a commercial PDK as [@XFAB2019]. \label{fig:LayerNumbers}]
\includegraphics{img/LayerColors.png}%[Checking layer collors in a commercial PDK as [@XFAB2019]. \label{fig:LayerColors}]

\begin{verbatim}
GDSNumber!Layer!Collor
4!DIFF!{rgb:red,0;green,255;blue,0}
5!POLY!{rgb:red,255;green,0;blue,0}
6!NIMP!{rgb:red,217;green,204;blue,0}
7!PIMP!{rgb:red,255;green,191;blue,242}
\end{verbatim}
\label{ver:LayerMAP}

To prove the advantage of using icLayoutRender tool, the authors have
rendered the IC layout known as StrongArm latched comparator and
proposed in (Fonseca et al., 2017). The original image file is
reproduced in Fig \autoref{fig:SA_Fonseca} using a grey color scale.
Figure \autoref{fig:SA_ColorRendering} depicts the same IC layout
rendered by the proposed tool in a colored version, and
\autoref{fig:SA_GrayRendering} is a gray scaled version of the
rendering. A full color version of this example is depicted in Fig.
\autoref{fig:SA_original} and \autoref{fig:SA_ColorRendering}. Another
example of IC layout known as OTA Miler is compared in Fig
\autoref{fig:OTA_original}, using standard tools, and in Fig
\autoref{fig:OTA_ColorRendering}, using the icLayoutRender solution.
Both illustration examples are obtained from an IC design using the PDK
of (XFAB Mixed-Signal Foundry Experts, 2019). The authors invite the
reader to now increase zoom over Fig. \autoref{fig:SA_original} and
\autoref{fig:SA_ColorRendering}, or over \autoref{fig:OTA_original} and
\autoref{fig:OTA_ColorRendering} to assert the vectorial graphical
rendering obtained by the proposal in which the standard tool is unable
to attain.

\includegraphics{img/SA_Fonseca.png}%[Original StrongArm latched comparator layout proposed in [@Fonseca2017]. \label{fig:SA_Fonseca}]
%\includegraphics{img/SA_GrayRendering.svg}%[StrongArm latched comparator layout rendered by the proposed tool (gray scaled version). \label{fig:SA_GrayRendering}]

\includegraphics{img/SA_original.png}%[Original StrongArm latched comparator layout in full color. \label{fig:SA_original}.]
%\includegraphics{img/SA_ColorRendering.svg}%[StrongArm latched comparator layout rendered by the proposed tool (colored version). \label{fig:SA_ColorRendering}]

\includegraphics{img/OTA_original.png}%Original OTA Miler amplifier layout.\label{fig:OTA_original}.
%\includegraphics{img/OTA_ColorRendering.svg}%[OTA Miler amplifier layout rendered by the proposed tool (colored version). \label{fig:OTA_ColorRendering}]

\hypertarget{references}{%
\section*{References}\label{references}}
\addcontentsline{toc}{section}{References}

\hypertarget{refs}{}
\begin{CSLReferences}{1}{0}
\leavevmode\hypertarget{ref-Calma1987}{}%
Calma Company. (1987). \emph{{GDSII™ Stream Format Manual}} (February).
\url{http://bitsavers.informatik.uni-stuttgart.de/pdf/calma/GDS_II_Stream_Format_Manual_6.0_Feb87.pdf}

\leavevmode\hypertarget{ref-Fonseca2017}{}%
Fonseca, A. V., Khattabi, R. E., Afshari, W. A., Barúqui, F. A. P.,
Soares, C. F. T., \& Ferreira, P. M. (2017). {A Temperature-Aware
Analysis of Latched Comparators for Smart Vehicle Applications}.
\emph{Proc ACM IEEE Symp. Integr. Circuits Syst. Design}, 1--6.
\url{https://doi.org/10.29292/jics.v13i1.8}

\leavevmode\hypertarget{ref-}{}%
Martins, J. R. O. R., Alves, F., \& Ferreira, P. M. (2021). {A 237 ppm /
◦ C L-Band Active Inductance Based Voltage Controlled Oscillator in SOI
0 . 18 µm}. \emph{Proc ACM IEEE Symp. Integr. Circuits Syst. Design},
1--6.

\leavevmode\hypertarget{ref-Vollmer2020}{}%
Vollmer, R. (2020). \emph{{GDSLatexConverter}}.
\url{https://github.com/Aypac/GDSLatexConverter}

\leavevmode\hypertarget{ref-XFAB2019}{}%
XFAB Mixed-Signal Foundry Experts. (2019). \emph{{XH018 - 0.18 Micron
Modular Analog Mixed HV Technology}} (pp. 1--22).
\url{https://www.xfab.com/technology/cmos/018-um-xh018/}

\end{CSLReferences}

\end{document}
